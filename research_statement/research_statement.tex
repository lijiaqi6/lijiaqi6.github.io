\documentclass[11pt,a4paper]{article}

% --- PACKAGES ---
\usepackage[margin=1in]{geometry}
\usepackage{times}
\usepackage{hyperref}
\usepackage{xcolor}
\usepackage{enumitem}
\usepackage{titlesec}

% --- BASIC SETUP ---
% Colors and hyperlinks
\definecolor{darkblue}{RGB}{0,0,0}
\hypersetup{
    colorlinks=true,
    linkcolor=darkblue,
    urlcolor=darkblue,
    citecolor=darkblue,
}

% Remove page numbers
\pagenumbering{gobble}

% --- SECTION FORMATTING ---
\titleformat{\section}{\large\bfseries\color{darkblue}}{}{0em}{}[\vspace{-0.3em}\hrule\vspace{0.5em}]
\titlespacing*{\section}{0pt}{1.5em}{0.5em}
\titlespacing*{\subsection}{0pt}{1em}{0.3em}

% --- DOCUMENT START ---
\begin{document}

% --- HEADER ---
\begin{center}
    {\Huge\bfseries RESEARCH STATEMENT} \\
    \vspace{0.5em}
    {\Large Jiaqi Li} \\
    \vspace{0.3em}
    Department of Economics, University of Warwick
\end{center}
\vspace{2em}

My research examines how household choices shape inequality across gender, race, and generations. I focus on how families allocate time between labor, leisure, and childrearing, and how these private decisions generate broader disparities in labor market outcomes and human capital accumulation.

A central theme in my work is that inequality emerges from household choices interacting with their institutional and economic environment. Mothers' labor supply after childbirth, fathers' involvement in parenting, and children's educational investments are not made in isolation: they are conditioned by economic forces, risks, and policies. These conditions shape both the opportunities available to families and the trade-offs they face. I work on both richer and poorer countries, which allows me to compare how households respond under very different institutional and economic settings.

My job market paper speaks to the puzzle of why Black women in the United States exhibit higher labour-market attachment than White women with similar observables. High-wage Black women return to work sooner after childbirth, incurring a smaller ``child penalty.'' I show that this is not simply a matter of preferences, but reflects heightened risks in the labor and marriage markets. Using a structural life-cycle model estimated on PSID data, I show that high-ability Black women rely on labour market attachment as self-insurance, sacrificing leisure while preserving parenting time. This generates smaller observed penalties in wages and employment but larger unobserved penalties via lost leisure. Counterfactuals reveal that equalising divorce and layoff risks can close up to 75\% of the racial gap in child penalties. I complement the model with quasi-experimental evidence: exploiting the staggered introduction of unilateral divorce reforms across U.S. states, I use a triple-difference design to show that Black women giving birth after these reforms experienced markedly smaller child penalties in employment, particularly in states where asset-division rules exposed women to greater financial risk. A falsification test on single women, who were not exposed to marital dissolution risks, shows no effect—reinforcing that the results reflect marital institutions rather than broader economic trends.

In a second paper, I turn to fathers. I show that heterogeneity in paternal time is a critical but overlooked driver of inequality in children's human capital. Using the PSID, I estimate a dynamic factor model of child development and find that equalising fathers' time across childhood stages would substantially reduce disparities—by 22\% in cognition and 49\% in health outcomes by adolescence. This highlights how differences in household choices, not just material resources, contribute to inequality.

A third paper studies children in Ethiopia. I show that road infrastructure critically shapes how rainfall shocks affect education. While existing work emphasises child labour as the main channel, I highlight commuting as an additional mechanism. Using panel data from 2002–2016, I document that excessive rainfall increases commuting time and reduces enrolment, but only in villages reliant on dirt rather than gravel roads. Event-study evidence and a cognition production function with time inputs show that commuting affects both the quantity and quality of study hours, with long commutes lowering effective learning. These findings demonstrate how infrastructure can buffer or amplify inequality in human capital.

Looking forward, I am interested in understanding how policies and economic forces reshape household choices and welfare across diverse contexts. My broader vision is to build a framework that connects family behavior, institutional change, and inequality, with the goal of illuminating how policy can expand opportunities across generations.

\vfill
\begin{center}
    \small Last updated: \today
\end{center}

\end{document}
