\documentclass[11pt,a4paper]{article}

% --- PACKAGES ---
\usepackage[margin=1in]{geometry}
\usepackage{times}
\usepackage{hyperref}
\usepackage{xcolor}
\usepackage{enumitem}
\usepackage{titlesec}

% --- BASIC SETUP ---
% Colors and hyperlinks
\definecolor{darkblue}{RGB}{0,0,0}
\hypersetup{
    colorlinks=true,
    linkcolor=darkblue,
    urlcolor=darkblue,
    citecolor=darkblue,
}

% Remove page numbers
\pagenumbering{gobble}

% --- SECTION FORMATTING ---
\titleformat{\section}{\large\bfseries\color{darkblue}}{}{0em}{}[\vspace{-0.3em}\hrule\vspace{0.5em}]
\titlespacing*{\section}{0pt}{1.5em}{0.5em}
\titlespacing*{\subsection}{0pt}{1em}{0.3em}

% --- DOCUMENT START ---
\begin{document}

% --- HEADER ---
\begin{center}
    {\Huge\bfseries RESEARCH STATEMENT} \\
    \vspace{0.5em}
    {\Large Jiaqi Li} \\
    \vspace{0.3em}
    Department of Economics, University of Warwick
\end{center}
\vspace{2em}

My research lies at the intersection of labor economics, gender economics, and applied microeconomics. I study how \textbf{institutional risks, legislation, and economic forces shape household choices and individual welfare}. By combining structural life-cycle models, rich longitudinal data, and quasi-experimental evidence, I aim to provide a deeper understanding of how families navigate uncertainty in labor and marriage markets, and how policies influence inequality across gender and race.

\section{Job Market Paper: Risks, Resilience, and Racial Gaps in Child Penalties}

Black women in the United States exhibit higher labor-market attachment than White women with similar observable characteristics. High-wage Black women, in particular, return to work sooner after childbirth and experience smaller observed ``child penalties.'' My job market paper seeks to explain this puzzle.

I show that heightened risks in the labor market (layoffs) and the marriage market (divorce and asset division) drive these differences. Using the Panel Study of Income Dynamics (PSID), I estimate a structural life-cycle model in which women allocate time between labor, leisure, and parenting under uncertainty. The results reveal that high-ability Black women increase labor market attachment as a form of self-insurance: they reduce leisure but preserve parenting time. This generates smaller observed penalties in wages and employment, but larger unobserved costs through lost leisure.

I complement the structural model with quasi-experimental evidence. Exploiting the staggered introduction of unilateral divorce reforms across U.S. states, I use a triple-difference strategy to show that Black women giving birth after these reforms faced markedly smaller child penalties in employment—particularly in title-based states where women bore greater financial risk. A falsification test on single women, who were not exposed to marital dissolution risks, finds no effect, confirming that the results reflect marital institutions rather than broader economic trends.

Counterfactual simulations suggest that equalizing divorce and layoff risks could close up to 75\% of the racial gap in child penalties. Taken together, the structural and reduced-form evidence highlight how institutional risks shape household decisions, and how reforms in family law and labor protection can substantially alter racial and gender inequalities.

\section{Other Research}

\subsection{Fathers' Time and Inequality in Child Outcomes}

In my second dissertation chapter, I turn to fathers. I show that heterogeneity in fathers' parenting time is a critical but overlooked driver of inequality in children's human capital. Using the PSID, I estimate a dynamic factor model of child development with a control-function approach, instrumenting paternal time with gender-specific local labor demand shocks. I find that equalizing fathers' time across childhood stages would substantially reduce disparities—by 22\% in cognition and 49\% in health outcomes by adolescence. These results highlight the central role of paternal involvement in shaping inequality and inform debates on parental leave and workplace flexibility.

\subsection{Infrastructure, Rainfall Shocks, and Education in Ethiopia}

My third chapter examines the interaction between infrastructure and climate shocks. Using Ethiopian panel data from 2002–2016, I study how road quality conditions the effect of rainfall on schooling. Excessive rainfall increases commuting time and reduces enrolment, but only in villages reliant on dirt roads. Event-study evidence and a cognition production function with time inputs show that commuting affects both the quantity and quality of study hours, with long commutes lowering effective learning. These results underscore how infrastructure investment can mitigate the educational costs of environmental risk.

\section{Future Research}

Looking ahead, I plan to extend my structural estimation work to the African context, where high-quality longitudinal data provides new opportunities to study how households adapt to risk. A central theme will be the role of legislative change and economic forces in shaping household choices and individual welfare. My broader agenda is to combine structural models with quasi-experimental evidence to understand how family dynamics and labor markets interact with institutions, with the goal of informing both academic debates and policy design.

\vfill
\begin{center}
    \small Last updated: \today
\end{center}

\end{document}
